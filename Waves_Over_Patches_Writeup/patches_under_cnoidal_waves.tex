 \documentclass[a4paper,11pt]{article}

\usepackage{amsmath}
\usepackage{amssymb}
\usepackage{amsthm}
\usepackage{graphicx}
\usepackage{caption}
\usepackage{subcaption}

\newtheorem{thm}{Theorem}
\newtheorem{lem}{Lemma}

\newcommand{\beq}{\begin{equation}}
\newcommand{\eeq}{\end{equation}}

\newcommand{\ba}{\begin{array}}
\newcommand{\ea}{\end{array}}

\newcommand{\bea}{\begin{eqnarray}}
\newcommand{\eea}{\end{eqnarray}}

\newcommand{\bc}{\begin{center}}
\newcommand{\ec}{\end{center}}

\newcommand{\ds}{\displaystyle}

\newcommand{\bt}{\begin{tabular}}
\newcommand{\et}{\end{tabular}}

\newcommand{\bi}{\begin{itemize}}
\newcommand{\ei}{\end{itemize}}

\newcommand{\bd}{\begin{description}}
\newcommand{\ed}{\end{description}}

\newcommand{\bp}{\begin{pmatrix}}
\newcommand{\ep}{\end{pmatrix}}

\newcommand{\p}{\partial}
\newcommand{\sech}{\mbox{sech}}

\newcommand{\cf}{{\it cf.}~}

\newcommand{\ltwo}{L_{2}(\mathbb{R}^{2})}
\newcommand{\smooth}{C^{\infty}_{0}(\mathbb{R}^{2})}

\newcommand{\br}{{\bf r}}
\newcommand{\bk}{{\bf k}}
\newcommand{\bv}{{\bf v}}

\setlength{\textheight}{212mm}
\setlength{\textwidth}{165mm}
\topmargin -6mm
\oddsidemargin -6mm


\newcommand{\gnorm}[1]{\left|\left| #1\right|\right|}
\newcommand{\ipro}[2]{\left<#1,#2 \right>}
\title{Vortex Patches under Cnoidal Waves}
\author{Christopher W. Curtis\\
Henrik Kalisch}
\date{}
\begin{document}
\maketitle
\section*{Introduction}
\section*{Model}
Throughout, we are attempting to describe the simultaneous evolution of a free surface $y = \eta(x,t) + H$, and a compactly supported patch of vorticity $\omega(x,y,t)$ underneath the free surface.  We suppose along the curve $z=0$ that we have a solid boundary so that the normal velocity is identically zero.  In an inviscid, incompressible fluid, we can represent the fluid velocity ${\bf u}(x,y,t)$ generated by a vortex patch characterized by vorticity profile $\omega({\bf x},t)$ over the compact domain $\Omega(t)$ via the integral equation
\[
{\bf u}({\bf x},t) = \int_{\Omega(t)} {\bf K}({\bf x}-\tilde{{\bf x}})\omega(\tilde{{\bf x}},t)d\tilde{{\bf x}} + \nabla \tilde{\phi}, ~ \Delta \tilde{\phi} = 0.
\]
where $\omega$ is the vorticity, and ${\bf K}$ is the standard Biot-Savart law kernel.  The harmonic function $\tilde{\phi}$ is used to address boundary conditions as explained in \cite{saffman}.  An attractive means for discretizing this equation as summarized in \cite{cottet} is to approximate the vorticity $\omega$ by a collection of $N$ point-vortices at positions ${\bf x}_{l}(t)$ via the expansion
\begin{equation}
\omega(\tilde{{\bf x}},t) = \sum_{j=1}^{N} \frac{\Gamma_{j}}{\delta^{2}}\chi\left(\frac{\tilde{{\bf x}}-{\bf x}_{l}(t)}{\delta}\right), ~ {\bf x}_{l}(t) = \left(x_{l}(t),y_{l}(t) \right),
\label{discvort} 
\end{equation}
where $\chi$ is some appropriately chosen mollifier, see \cite{beale}, and $\Gamma_{j}$ is the circulation associated with the point vortex at ${\bf x}_{l}(t)$.  Thus, we can reduce the problem of tracking the evolution of the vortex patch to describing the motion of the point vortices via the system of ODE's
\[
\frac{d{\bf x}_{j}}{dt}  =  \sum_{l\neq j}^{N} \Gamma_{l} {\bf K}_{\delta}\left({\bf x}_{j}-{\bf x}_{l}\right) + \nabla \tilde{\phi}\left({\bf x}_{j},t\right), ~ {\bf K}_{\delta}({\bf x}) = \frac{1}{\delta^{2}}\int_{\mathbb{R}^{2}}{\bf K}({\bf x} - \tilde{\bf x})\chi \left(\frac{\tilde{{\bf x}}}{\delta} \right) d \tilde{{\bf x}}.
\]
Choosing, as in \cite{beale}, the mollifier $\chi$ to be the fourth-order kernel 
\[
\chi(r) = 2e^{-r^{2}} - \frac{1}{2}e^{-r^2/2}, 
\]
introducing periodic boundary conditions in the lateral direction and a solid boundary along the curve $z=0$ then modifies the above dynamical system to be 
\[
i\frac{d z^{\ast}_{j}}{dt}  = \frac{1}{2\pi}\left(\sum_{l\neq j}^{N} \Gamma_{l} \sum_{m=-\infty}^{\infty} \frac{\tilde{\chi}(z_{j}-z_{l}-2Lm;\delta)}{z_{j}-z_{l}-2Lm} -   \sum_{l=1}^{N} \Gamma_{l} \sum_{m=-\infty}^{\infty} \frac{\tilde{\chi}(z_{j}-z^{\ast}_{l}-2Lm;\delta)}{z_{j}-z^{\ast}_{l}-2Lm}\right)  + \p_{y}\tilde{\phi} + i\p_{x}\tilde{\phi}, 
\]
where $z_{j}=x_{j} + iy_{j}$, the period in $x$ is given by $2L$, and
\[
\tilde{\chi}(r;\delta) = \left(1-e^{-r^{2}/2\delta^{2}} \right)\left(1+2e^{-r^{2}/2\delta^{2}} \right).
\]

As can be seen, the presence of the mollifier prevents from the closed form evaluation of the sums in $m$, thereby potentially adding significant overhead in numerical computations, even if fast Fourier transforms are used to evaluate the sums.  We note however that 
\[
\tilde{\chi}(r;\delta) = 1 + \bar{\chi}(r), ~ \bar{\chi}(r) = \left(1-2e^{-r^2/2\delta^{2}} \right)e^{-r^2/2\delta^{2}}
\]
which tacitly explains the role of mollifcation, which is to remove singularities in the determination of particular velocities when $\left|z_{j}-z_{l} \right|\lesssim \delta$.  Thus, when we know that $\left|z_{j}-z_{l} \right| > \delta$, we take $\tilde{\chi}(r;\delta) \sim 1$ so that 
\[
\frac{1}{2\pi}\sum_{m=-\infty}^{\infty} \frac{\tilde{\chi}(z_{j}-z_{l}-2Lm;\delta)}{z_{j}-z_{l}-2Lm} \approx \frac{1}{4L}\cot\left(\frac{\pi}{2L}\left(z_{j}-z_{l}\right) \right),
\]
where the sum is taken in the principal value sense.  In the case that $\left|z_{j}-z_{l} \right|\lesssim \delta$, we use instead 
\[
\frac{1}{2\pi}\sum_{m=-\infty}^{\infty} \frac{\tilde{\chi}(z_{j}-z_{l}-2Lm;\delta)}{z_{j}-z_{l}-2Lm} \approx \frac{1}{4L}\cot\left(\frac{\pi}{2L}\left(z_{j}-z_{l}\right) \right) + \frac{1}{2\pi}\frac{\bar{\chi}(z_{j}-z_{l};\delta)}{z_{j}-z_{l}}.
\]
The error incurred in these approximations is only exponentially small.  We evalute the corresponding sums over the image points $z_{j}-z_{l}^{\ast}$ so as to keep the zero flow through $z=0$ condition strictly enforced.  Our use of a Fast-Multipole Method for the evaluation of the velocities $\dot{z}_{j}$ in effect determines all points either far or close to $z_{j}$, and thus the approximation above is a very natural and easy one to use in our numerical scheme.  

Following the arguments in \cite{curtis}, and again emphasizing the compact support of the vorticity $\omega(x,y,t)$, we then have at the free surface the coupled nonlinear system
\[
\eta_{t} = -\p_{x}\eta\p_{x}\tilde{\phi} + \p_{z}\tilde{\phi} + P_{v},
\]
and
\begin{multline*}
\tilde{\phi}_{t} + \frac{1}{2}\left|\nabla \tilde{\phi}\right|^{2} +\mbox{Im}\left\{Q_{v}\right\}\p_{x}\tilde{\phi} + \mbox{Re}\left\{Q_{v}\right\}\p_{z}\tilde{\phi} + g\eta = E_{v} - \frac{1}{2}\left|Q_{v}\right|^{2} + \frac{\sigma}{\rho_{0}}\p_{x}\left(\frac{\p_{x}\eta}{\sqrt{1+(\p_{x}\eta)^{2}}} \right)
\end{multline*}
where we have defined
\[
c(\eta,z_{j}) = \cot\left(\frac{\pi}{2L}\left(\eta+H-z_{j}\right)\right),
\]
so that 
\[
P_{v} = \mbox{Re}\left\{Q_{v}\right\} - \mbox{Im}\left\{Q_{v}\right\}\p_{x}\eta , 
\]
\[
Q_{v} = \frac{1}{4L}\sum_{j=1}^{N}\Gamma_{j}\left(c(\eta,z_{j}) - c(\eta,z^{\ast}_{j})\right),
\]
and
\[
E_{v} = \frac{1}{4L}\sum_{j=1}^{N}\Gamma_{j}\left(\dot{x}_{j}\mbox{Im}\left\{c(\eta,z_{j})-c(\eta,z^{\ast}_{j})\right\} + \dot{z}_{j}\mbox{Re}\left\{c(\eta,z_{j})+c(\eta,z^{\ast}_{j})\right\}\right)
\]
Note, we have ignored the mollification given the seperation between the surface and the point vortices used to approximate the vortex patch.  

Defining $q = \tilde{\phi}|_{z=\eta+H}$, standard arguments \cite{craig,curtis} allow for the derivation of series representations to the Dirichlet-to-Nuemann operator $G(\eta)$ so that 
\[
\eta_{t} = G(\eta)q + P_{v},
\]
and 
\begin{multline*}
q_{t} + \frac{1}{2}\left(\p_{x}q\right)^{2} + g\eta - E_{v} + \frac{1}{2}\left|Q_{v}\right|^{2} - \frac{\sigma}{\rho_{0}}\p_{x}\left(\frac{\p_{x}\eta}{\sqrt{1+(\p_{x}\eta)^{2}}} \right)=\\
- \frac{1}{1+(\p_{x}\eta)^{2}}\left(\left(P_{v}+\mbox{Re}\left\{Q_{v}\right\}-\frac{1}{2}\left(Gq+\p_{x}\eta\p_{x}q\right)\right)\left(Gq+\p_{x}\eta\p_{x}q\right) + \mbox{Im}\left\{Q_{v}\right\}(\p_{x}q - \p_{x}\eta Gq) \right) 
\end{multline*}
Thus, the surface boundary conditions can be recast entirely in terms of surface variables alone.  This then leaves the problem of evaluating the derivatives of $\tilde{\phi}$ at the vortex positions thereby allowing us to computing the speeds of the point vortices and closing the system of equations in terms of $\eta$, $q$, and $z_{j}$.  To do this, we repeat the arguments in \cite{curtis}, where it was shown that 
\[
\left. \p_{y}\tilde{\phi} + i\p_{x}\tilde{\phi}\right|_{z_{j}} = -\frac{1}{4L}\int_{-L}^{L}\left((c(\eta,z_{j})-c^{\ast}(\eta,z^{\ast}_{j}))\p_{x}q - i(c(\eta,z_{j})+c^{\ast}(\eta,z^{\ast}_{j}))G(\eta)q \right)dx
\]
\section*{Results}
\section*{Conclusion}
\section*{Appendix}
\bibliography{waves_over_vortices}
\bibliographystyle{unsrt}
\end{document}