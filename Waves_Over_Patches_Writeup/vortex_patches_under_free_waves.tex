 \documentclass[a4paper,11pt]{article}

\usepackage{amsmath}
\usepackage{amssymb}
\usepackage{amsthm}
\usepackage{graphicx}
\usepackage{caption}
\usepackage{subcaption}

\newtheorem{thm}{Theorem}
\newtheorem{lem}{Lemma}

\newcommand{\beq}{\begin{equation}}
\newcommand{\eeq}{\end{equation}}

\newcommand{\ba}{\begin{array}}
\newcommand{\ea}{\end{array}}

\newcommand{\bea}{\begin{eqnarray}}
\newcommand{\eea}{\end{eqnarray}}

\newcommand{\bc}{\begin{center}}
\newcommand{\ec}{\end{center}}

\newcommand{\ds}{\displaystyle}

\newcommand{\bt}{\begin{tabular}}
\newcommand{\et}{\end{tabular}}

\newcommand{\bi}{\begin{itemize}}
\newcommand{\ei}{\end{itemize}}

\newcommand{\bd}{\begin{description}}
\newcommand{\ed}{\end{description}}

\newcommand{\bp}{\begin{pmatrix}}
\newcommand{\ep}{\end{pmatrix}}

\newcommand{\p}{\partial}
\newcommand{\sech}{\mbox{sech}}

\newcommand{\cf}{{\it cf.}~}

\newcommand{\ltwo}{L_{2}(\mathbb{R}^{2})}
\newcommand{\smooth}{C^{\infty}_{0}(\mathbb{R}^{2})}

\newcommand{\br}{{\bf r}}
\newcommand{\bk}{{\bf k}}
\newcommand{\bv}{{\bf v}}

\setlength{\textheight}{212mm}
\setlength{\textwidth}{165mm}
\topmargin -6mm
\oddsidemargin -6mm


\newcommand{\gnorm}[1]{\left|\left| #1\right|\right|}
\newcommand{\ipro}[2]{\left<#1,#2 \right>}
\title{Vortex Patches Under Free Surface Waves}
\date{}
\begin{document}
\maketitle
\section{Model Setup}
In an inviscid, incompressible fluid, we can generally represent the fluid velocity ${\bf u}(x,z,t)$ via the integral equation
\[
{\bf u}(x,z,t) = \int_{\Omega(t)} {\bf K}(x-\tilde{x},z-\tilde{z})\omega(\tilde{x},\tilde{z},t)d\tilde{x}d\tilde{z} + \nabla \tilde{\phi}, ~ \Delta \tilde{\phi} = 0.
\]
where $\omega$ is the vorticity, and ${\bf K}$ is the standard Biot-Savart law style kernel.  

However, the presenence of a freely evolving surface makes the complete description of the fluid in this case difficult.  Thus, in \cite{curtis}, building on the methodology in \cite{afm}, we develop a model which describes the evolution of a free surface $z=\eta(x,t)$ moving over a collection of $N$ 
\section{Validation}

\section{Numerical Experiments}

\bibliography{waves_over_vortices}
\end{document}