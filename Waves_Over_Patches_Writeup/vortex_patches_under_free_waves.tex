 \documentclass[a4paper,11pt]{article}

\usepackage{amsmath}
\usepackage{amssymb}
\usepackage{amsthm}
\usepackage{graphicx}
\usepackage{caption}
\usepackage{subcaption}

\newtheorem{thm}{Theorem}
\newtheorem{lem}{Lemma}

\newcommand{\beq}{\begin{equation}}
\newcommand{\eeq}{\end{equation}}

\newcommand{\ba}{\begin{array}}
\newcommand{\ea}{\end{array}}

\newcommand{\bea}{\begin{eqnarray}}
\newcommand{\eea}{\end{eqnarray}}

\newcommand{\bc}{\begin{center}}
\newcommand{\ec}{\end{center}}

\newcommand{\ds}{\displaystyle}

\newcommand{\bt}{\begin{tabular}}
\newcommand{\et}{\end{tabular}}

\newcommand{\bi}{\begin{itemize}}
\newcommand{\ei}{\end{itemize}}

\newcommand{\bd}{\begin{description}}
\newcommand{\ed}{\end{description}}

\newcommand{\bp}{\begin{pmatrix}}
\newcommand{\ep}{\end{pmatrix}}

\newcommand{\p}{\partial}
\newcommand{\sech}{\mbox{sech}}

\newcommand{\cf}{{\it cf.}~}

\newcommand{\ltwo}{L_{2}(\mathbb{R}^{2})}
\newcommand{\smooth}{C^{\infty}_{0}(\mathbb{R}^{2})}

\newcommand{\br}{{\bf r}}
\newcommand{\bk}{{\bf k}}
\newcommand{\bv}{{\bf v}}

\setlength{\textheight}{212mm}
\setlength{\textwidth}{165mm}
\topmargin -6mm
\oddsidemargin -6mm


\newcommand{\gnorm}[1]{\left|\left| #1\right|\right|}
\newcommand{\ipro}[2]{\left<#1,#2 \right>}
\title{Vortex Patches Under Free Surface Waves}
\date{}
\begin{document}
\maketitle
\begin{abstract}
In this paper, we examine the interactions of freely evolving surfaces in shallow-water scalings with elliptical vortex patches.   The impact of the waves on the stability of the patches is examined, and the manifestation of instabilities in the patches on key surface wave metrics, such as the associated wave spectrum, are studied. 
\end{abstract}
\section*{Model Setup}
In an inviscid, incompressible fluid, we can represent the fluid velocity ${\bf u}(x,z,t)$ generated by a vortex patch characterized by vorticity profile $\omega({\bf x},t)$ over the compact domain $\Omega(t)$ via the integral equation
\[
{\bf u}({\bf x},t) = \int_{\Omega(t)} {\bf K}({\bf x}-\tilde{{\bf x}})\omega(\tilde{{\bf x}},t)d\tilde{x}d\tilde{z} + \nabla \tilde{\phi}, ~ \Delta \tilde{\phi} = 0.
\]
where $\omega$ is the vorticity, and ${\bf K}$ is the standard Biot-Savart law kernel.  The harmonic function $\tilde{\phi}$ is used to address boundary conditions as explained in \cite{saffman}.  An attractive means for discretizing this equation as summarized in \cite{cottet} is to approximate the vorticity $\omega$ by a collection of $N$ point-vortices at positions ${\bf x}_{l}(t)$ via the expansion
\[
\omega(\tilde{{\bf x}},t) = \sum_{j=1}^{N} \frac{\Gamma_{j}}{\delta^{2}}\chi\left(\frac{\tilde{{\bf x}}-{\bf x}_{l}(t)}{\delta}\right) 
\]
where $\chi$ is some appropriately chosen mollifier, see \cite{beale}, and $\Gamma_{j}$ is the circulation associated with the point vortex at ${\bf x}_{l}(t)$.  Thus, we can reduce the problem of tracking the evolution of the vortex patch to describing the motion of the point vortices via the system of ODE's
\[
\frac{d{\bf x}_{j}}{dt}  =  \sum_{l\neq j}^{N} \Gamma_{l} {\bf K}_{\delta}\left({\bf x}_{j}-{\bf x}_{l}\right) + \nabla \tilde{\phi}\left({\bf x}_{j},t\right), ~ {\bf K}_{\delta}({\bf x}) = \frac{1}{\delta^{2}}\int_{\mathbb{R}^{2}}{\bf K}({\bf x} - \tilde{\bf x})\chi \left(\frac{\tilde{{\bf x}}}{\delta} \right) d \tilde{{\bf x}}.
\]
The error in this approximation can be understood as a balance between the spacing of the point-vortices and the choice of mollifier, see the proofs in \cite{cottet}.  There, a vague notion of a ``particle overlap" critierion is introduced, which while somewhat conceptually pleasing, is difficult to apply in practice.  A more straightforward understanding of error for this method is by way of interpolation.  To wit, if we start with a regularly spaced array of points ${\bf x}_{l}(0)$ then as the flow evolves in time, shearing will distort this grid, clustering points in some regions and making them relatively sparse in others.  Thus, the error associated with using said points in an interpolatory scheme can become highly irregular, representing the overall error in the method.  

\subsection*{Free-Surface Modeling}
However, if the boundary we intend to include is a freely evolving surface, representing waves moving over a vortex patch, this makes the complete description of the fluid difficult.  Thus, in \cite{curtis}, building on the methodology in \cite{afm}, we developed a model which describes the evolution of a free surface $z=H + \eta(x,t)$ and the associated surface velocity $Q(x,t) = \p_{x} \tilde{\phi}(x,\eta(x,t),t)$ moving over a collection of $N$-point vortices.  To do so, we introduce the following non-dimensionalizations
\[
\tilde{x} = \frac{x}{L}, ~\tilde{z} = \frac{z}{H}, ~ \tilde{t} = \frac{\sqrt{gH}}{L} t, ~ \eta = d\tilde{\eta}, ~ \tilde{\phi} = \mu L\sqrt{gH} \tilde{\tilde{\phi}}, ~ \tilde{\Gamma}_{j} = \frac{\Gamma_{j}}{\Gamma},
\]
where we define the non-dimensional parameters
\[
\mu= \frac{d}{H}, ~ \gamma = \frac{H}{L}.  
\]
Note, in this scaling, we see that the vorticity $\omega$ is then scaled as 
\[
\omega = \frac{\mu \sqrt{gH}}{H}\tilde{\omega},
\]
so that by using Stoke's theorem, we see the net circulation $\Gamma$ can be written as 
\[
\Gamma = \mu L \sqrt{gH} \tilde{\Gamma}, ~ \tilde{\Gamma} = \int_{ \tilde{\Omega} } \tilde{\omega} d\tilde{A}.
\]
The rescaled net circulation on the system, $\tilde{\Gamma}$, is a constant of the flow due to Kelvin's theorem.  In these scalings, after dropping tildes, the surface is now at $z = 1 + \mu \eta(x,t)$,  so that by taking $\mu$ and $\gamma$ to be small, we are describing small amplitdue, long-wavelength waves.  

If we fix the bottom of our domain at $z=0$ to be a solid boundary, we note that we must modify the Biot-Savart kernel ${\bf K}$ via a method-of-images argument, so that we instead use the kernel $\tilde{{\bf K}}_{\delta}(x,z,\tilde{x},\tilde{z})$ where 
\[
\tilde{{\bf K}}_{\delta}(x,z,\tilde{x},\tilde{z}) = {\bf K}_{\delta}(x-\tilde{x},z-\tilde{z}) - {\bf K}_{\delta}(x-\tilde{x},z+\tilde{z}),
\]
so that by also requiring $\tilde{\phi}_{z}(x,0,t)=0$, we ensure there is zero flow through the solid boundary.  Likewise, we ultimately want to have periodic boundary conditions in $x$, that we ultimately need to evaluate the molified kernel $\tilde{{\bf K}}_{\delta,p}(x,z,\tilde{x},\tilde{z})$ where
\[
\tilde{{\bf K}}_{\delta,p}(x,z,\tilde{x},\tilde{z}) = \sum_{m=-\infty}^{\infty} \tilde{{\bf K}}_{\delta}(x-2mL,z,\tilde{x},\tilde{z}),
\]
where we take $2L$ to be the period of the simulation in unscaled coordinates.  
\section*{Validation}
In order to better understand the discretization error associated with the introduction of the mollification kernel, we use the Kirchoff elliptical vortices \cite{mitchell,crosby} as exact solutions to which we compare our numerical simulations.  We note in this section that the free surface $z=\mu\eta(x,t)$ is not included, and thus we ignore the harmonic function $\tilde{\phi}$.  In the shallow water scalings used above, a Kirchoff elliptical vortex is a vortex patch with constant vorticity $\omega$ and support over the ellipse
\[
\left(\frac{x}{a}\right)^{2} + \gamma^{2}\left(\frac{z}{b} \right)^{2} = 1.
\]
It is a now classical result, see \cite{mitchell} for details, that the patch simply rotates with $\dot{a}=\dot{b}=0$ and the angle describing the rotation being given in these coordinates by 
\[
\dot{\theta} = \frac{\mu}{\gamma} \frac{\tilde{\omega}ab}{(a+b)^{2}}.
\]
\section*{Numerical Experiments}

\bibliography{waves_over_vortices}
\bibliographystyle{unsrt}
\end{document}